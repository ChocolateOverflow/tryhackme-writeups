\chapter{Results}
In this chapter, the vulnerabilities found during the penetration test are presented. All the issues are grouped by target and contain the following information:
\begin{itemize}
	\item Brief description.
	\item CVSS Base Score -- see \href{https://www.first.org/cvss/user-guide}{\textcolor{blue}{\underline{here}}} for details.
	\item Exploitability -- describes the likelihood of an issue being used against customer's infrastructure.
	\item Business impact.
	\item References to classifications: WASC, OWASP, CWE.
	\item Steps to reproduce.
\end{itemize}

\newpage

\section{Public facing web server}
\textbf{Hostnames}: thomaswreath.thm, prod-serv\\
\textbf{Server IP address}: 10.200.177.200

This is the only public facing server among the targets. The page on ports 80 and 443, which reveal the hostname \lstinline{thomaswreath.thm}, are simply a static page with nothing exploitable. The web service on port 10000, however, is vulnerable to CVE-2019-15107 and is ran as root, thus giving me an easy exploitation to become root on the server.\\

\subsection{CVE-2019-15107: Command Injection} \label{ss:issue-1}
The version of Minserv (1.890) running on the server has a known vulnerability CVE-2019-15107, giving us code execution on the server. In addition, the server is running as root, thus giving us code execution as root.

Basic information about this issue is presented in Table \ref{tbl:issue-1}.
\begin{table}[h]
	\centering
	\begin{tabular}{| l | p{10cm} |}
		\hline
		Description & CVE-2019-15107: Command Injection as root \\
		\hline
		CVSS Base Score & 9.8 \\
		\hline
		Exploitablity & High \\
		\hline
		Business impact & Total control over server. \\
		\hline
    References to classifications & CWE-78, CWE-250 \\
		\hline
	\end{tabular}
\caption{Issue \#1: Command Injection as root}
\label{tbl:issue-1}
\end{table}

\subsubsection{Minimal proof of concept}
\begin{enumerate}
    \item Download PoC from https://github.com/MuirlandOracle/CVE-2019-15107
    \item run exploit: \lstinline{./CVE-2019-15107.py thomaswreath.thm}
\end{enumerate}

\subsubsection{Proposed solutions} \label{solution:issue-1}
Update Minserv

\newpage

\section{Git Server}
\textbf{Hostname}: git-serv\\
\textbf{Server IP address}: 10.200.177.150

This git server is internal and is said to contain code of Mr.Wreath's code for his website. The git server is running an old version of GitStack vulnerable to a well-known Unauthenticated Remote Code Execution exploit and is running the web service as "nt authority\textbackslash system".

\subsection{Unauthenticated Remote Code Execution} \label{ss:issue-2}
The version of GitStack running on the git server is vulnerable to Unauthenticated Remote Code Execution with a pre-made PoC. Additionally, the web server is running as "nt authority\textbackslash system", thus running the PoC immediately gives us code execution as "nt authority\textbackslash system" on git-serv.

Basic information about this issue is presented in Table \ref{tbl:issue-2}.
\begin{table}[h]
	\centering
	\begin{tabular}{| l | p{10cm} |}
		\hline
		Description & Unauthenticated Remote Code Execution as "nt-authority\textbackslash system" \\
		\hline
		CVSS Base Score & 10.0 \\
		\hline
		Exploitablity & High \\
		\hline
		Business impact & Complete control over server, source code leakage.  \\
		\hline
    References to classifications & CWE-78, CWE-250 \\
		\hline
	\end{tabular}
\caption{Issue \#2: Unauthenticated Remote Code Execution}
\label{tbl:issue-2}
\end{table}

\subsubsection{Minimal proof of concept}
\begin{enumerate}
  \item Download exploit: \href{https://www.exploit-db.com/exploits/43777}{https://www.exploit-db.com/exploits/43777}
  \item Modify explioit to change backdoor location (modified exploit in section \ref{appendix-1}, page \pageref{appendix-1})
  \item Run exploit
  \item Acquire and use backdoor at \lstinline{/web/exploit-chocola.php}
\end{enumerate}

\subsubsection{Proposed solutions} \label{solution:issue-2}
Update GitStack

\newpage

\section{Personal PC}
\textbf{Hostname}: wreath-pc\\
\textbf{Server IP address}: 10.200.177.100
The web service running on port 80 uses the source code found in git-serv. Analyzing the source code, we're able to identified a flawed filter in the file upload functionality and abuse it to upload PHP code, giving us code execution. With a shell on wreath-pc, we find that the service "SystemExplorerHelpService" is vulnerable to "SystemExplorerHelpService", which we use to become "nt authority\textbackslash system" on wreath-pc.

\subsection{Arbitrary File Upload} \label{ss:issue-3}
The file upload filter checks the 2nd instead of the last extension in the file name of the uploaded file, making it possible to run PHP code by uploading a file whose 2nd extension is that of an image and last extension is ".php".

Basic information about this issue is presented in Table \ref{tbl:issue-3}.
\begin{table}[h]
	\centering
	\begin{tabular}{| l | p{10cm} |}
		\hline
		Description & Insufficient validation of uploaded files allow for upload of PHP files, leading to execution of arbitrary PHP code on the server \\
		\hline
		CVSS Base Score & 6.8 \\
		\hline
		Exploitablity & Medium \\
		\hline
		Business impact & Total compromise of source code, no immediate impact on availability.  \\
		\hline
		References to classifications & CWE-434, CWE-646 \\
		\hline
	\end{tabular}
	\caption{Issue \#3: Improper Validation of Uploaded Files}
	\label{tbl:issue-3}
\end{table}

\subsubsection{Minimal proof of concept}
\begin{enumerate}
  \item Create a valid image file (e.g. PNG)
  \item Create a one-line PHP code to be executed
  \item Embed PHP code in image: \lstinline{exiftool -Comment=<PHP one-liner> filename.png.php}
  \item Upload malicious image file
  \item Goto \lstinline{/resources/uploads/filename.png.php} to execute PHP code
\end{enumerate}

\subsubsection{Proposed solutions} \label{solution:issue-3}
Implement stricter file validation, checking the last file extension instead of the 2nd.

\subsection{Unquoted Service Path} \label{ss:issue-4}
Ther service "SystemExplorerHelpService" has its path unquoted and with spaces. Additionally, a directory in the unquoted path is given "FullControl" access the our user, which is excessive privilege. Together, they make it possible to trick Windows into running a malicious file elsewher in the path

Basic information about this issue is presented in Table \ref{tbl:issue-4}.
\begin{table}[h]
	\centering
	\begin{tabular}{| l | p{10cm} |}
		\hline
		Description & Unquoted Path of the Service "SystemExplorerHelpService" allows escalation to the user "nt-authority\textbackslash system" \\
		\hline
		CVSS Base Score & 7.8 \\
		\hline
		Exploitablity & High \\
		\hline
		Business impact & Total control over machine \\
		\hline
		References to classifications & CWE-428 \\
		\hline
	\end{tabular}
	\caption{Issue \#4: Unquoted Service Path}
	\label{tbl:issue-4}
\end{table}

\subsubsection{Minimal proof of concept}
\begin{enumerate}
  \item Write program to execute desired code (code used is in section \ref{appendix-2}, page \pageref{appendix-2})
  \item Place compiled binary in "C:\textbackslash Program Files (x86)\textbackslash System Explorer\textbackslash System.exe"
  \item Restart the service "SystemExplorerHelpService" (stop/start or wait until machine is restarted)
\end{enumerate}

\subsubsection{Proposed solutions} \label{solution:issue-4}
Change the service to use the fully quoted path.
